When it comes to interpreting real estate market dynamics, drawing conclusions from data can be more of an art than a science.\cite{Brett2009}

\subsection{The study of complex urban systems, modelling and computation} \label{subsec:study_of_complex_urban_systems_modelling_computation}

An increasing amount of aspects of human life can be traced back through diverse digital footprints and, when aggregated, can reveal emerging patterns.\cite{Arribas-Bel2014}
As such forms of human activity as transactions of land and real estate ownership become digitized\cite{TeranetEnterprisesInc.}, a wealth of data becomes captured and available for analysis.
These increasingly comprehensive archives of human behaviour, combined with the exponential growth of computational power, create potential for transformations in such fields as sociology\cite{Lazer2017} and travel behaviour analysis\cite{Chen2016}.

However, when it comes to using these data sources in social studies, along with opportunities there are also challenges present.
For example, these data sources can have issues with the quality of the data, might require a specific set of skills to take advantage of these data sources, or might not be suitable for traditional methods meant for traditional data\cite{Arribas-Bel2014}.
In addition, when it comes to interpreting real estate market dynamics, drawing conclusions from data can be more of an art than a science.\cite{Brett2009}
Therefore, to study the interaction between land development, urban form, and transportation, it would be beneficial to have a system that can facilitate efficient access to linked information describing these phenomena to researchers from a wide range of disciplines and backgrounds.

The study of complex systems, including urban systems such as land development or transportation, is intrinsically tied to modelling and computation.
Various computer-based models allow us to explore and improve our understanding of the behaviour of a complex system.
Thus, systematic analysis and modelling of urban systems intrinsically depends on our ability to represent and model urban regions and urban processes, which in turn is often limited by our computing software and hardware capabilities.

With large amounts of information about urban regions digitalized\cite{TeranetEnterprisesInc.}, computer-based data storage and data display, manipulation, and management systems, such as Geographic Information Systems (GIS) or Relational Database Management Systems (RDBMS), nowadays play an increasingly important role.
These systems become important because they offer researchers easy access to a wide array of diverse data sources and allow efficient workflows to prepare datasets to be used in statistical analysis software, modelling software, machine learning, etc.

\subsection{Real estate market} \label{subsec:real_estate_market}

In speaking of trade we always explicitly speaking of the concept of a market.
Markets, such as the land and real estate markets, are fundamental to understanding of most spatial socio-economic processes that drive the city's evolution.
There are many ways to approach the definition of the real estate market\cite{Brett2009}: it could refer to various grouping of customers, as often done in marketing context, while terms supply and demand employed by economists characterize the market in terms of by analyzing behaviour of buyers and sellers.
In real estate, product usually refers to property type, and is distinguished by the property type (apartment buildings, offices, warehouses, etc.), interior features (size, layout, quality of design, amenities/services), location attributes (walkability, access to transit, distance to the city centre, etc.), and prices and rents.
Segmentation of the housing sector can be performed by physical characteristics (single family detached/attached/low-rise/mid-rise/high-rise apartments, etc.).
Narrow definition of the market segment helps fine tune the analysis.
As such, it would be beneficial to relate a range of data sources together into a single database that can yield targeted datasets with custom attributes to facilitate specialized studies.
This task, which involves the collection of data, design, and implementation of a relational database covering the GTHA housing market, is the primary focus of this master's thesis.

TODO: fuzzy diagram of price

