When it comes to interpreting real estate market dynamics, drawing conclusions from data can be more of an art than a science.\cite{Brett2009}

\subsection{The study of complex urban systems, modelling and computation} \label{subsec:study_of_complex_urban_systems_modelling_computation}

An increasing amount of aspects of human life can be traced back through diverse digital footprints and, when aggregated, can reveal emerging patterns.\cite{Arribas-Bel2014}
As such forms of human activity as transactions of land and real estate ownership become digitized\cite{TeranetEnterprisesInc.}, a wealth of data becomes captured and available for analysis.
These increasingly comprehensive archives of human behaviour, combined with the exponential growth of computational power, create potential for transformations in such fields as sociology\cite{Lazer2017} and travel behaviour analysis\cite{Chen2016}.

However, when it comes to using these data sources in social studies, along with opportunities there are also challenges present.
For example, these data sources can have issues with the quality of the data, might require a specific set of skills to take advantage of these data sources, or might not be suitable for traditional methods meant for traditional data\cite{Arribas-Bel2014}.
In addition, when it comes to interpreting real estate market dynamics, drawing conclusions from data can be more of an art than a science.\cite{Brett2009}
Therefore, to study the interaction between land development, urban form, and transportation, it would be beneficial to have a system that can facilitate efficient access to linked information describing these phenomena to researchers from a wide range of disciplines and backgrounds.

The study of complex systems, including urban systems such as land development or transportation, is intrinsically tied to modelling and computation.
Various computer-based models allow us to explore and improve our understanding of the behaviour of a complex system.
Thus, systematic analysis and modelling of urban systems intrinsically depends on our ability to represent and model urban regions and urban processes, which in turn is often limited by our computing software and hardware capabilities.

With large amounts of information about urban regions digitalized\cite{TeranetEnterprisesInc.}, computer-based data storage and data display, manipulation, and management systems, such as Geographic Information Systems (GIS) or Relational Database Management Systems (RDBMS), nowadays play an increasingly important role.
These systems become important because they offer researchers easy access to a wide array of diverse data sources and allow efficient workflows to prepare datasets to be used in statistical analysis software, modelling software, machine learning, etc.


TODO: fuzzy diagram of price

Bourne\cite{Bourne1982} defines urban structure as the combination of the following elements:
\begin{itemize}
    \item urban form, or the spatial configuration of fixed elements within the urban region (including land use, buildings, transportation network, etc.)
    \item urban interactions, or the flows of goods, people, information, and money (including commercial activity, real estate market, etc.)
    \item organizing principles, or the relationships between the urban form and interactions (such as travel cost minimization, social status, segregation, planning policy, etc.)
\end{itemize}


One of the possible ways of studying the interaction between the housing market, urban form, and transportation is empirical, such as examining how property values vary with the distance to a transportation facility\cite{Sherry1999}.
However, this type of research requires fine-scale transportation data to be linked to fine-scale housing market data, which presents a challenge since these data sources use different spatial units (as discussed in section~\ref{sec:spatial_relationships}) and are available at different temporal spans (as discussed in section~\ref{sec:termporal_relationships_between_datasets}).
The primary focus of this mater's thesis is the introduction of a knowledge system providing researchers at the University of Toronto Transportation Research Institute (UTTRI) access to the information required to investigate the complex relationship between transportation and land use.

\subsection{Housing market} \label{subsec:housing_market}

\subsection{Models used for transportation and land use} \label{subsec:transportation_land_use_interaction}


Iacono et al.
\cite{Iacono2008} provide an overview of some of the most common frameworks of Land Use and Transport (LUT) models that have been used to model interaction between transportation and land use in urbanized regions.
Despite the difficulty of modelling of every relevant aspect of an urban region, a wide variety of models were produced dealing with the relationship between transportation network growth and changes in land use and the location of economic activity.
The frameworks can broadly be broken into two major approaches to modelling interactions between land use and transport:
\begin{itemize}
    \item "top-down" modeling frameworks, where interaction between transportation networks and location is specified as a set of aggregate relationships.
    These relationships are based on the behaviour of a representative individual, and are usually taken as a mean calculated from a representative sample of the population.
    These models include:
    \begin{itemize}
        \item aggregate models of spatial interaction
        \item econometric models
    \end{itemize}
    \item "bottom-up" microsimulation models, which cover a number of different approaches to representing the dynamics of land use change and travel behaviour through disaggregating the population and simulating the changes.
    These models include:
    \begin{itemize}
        \item activity-based travel models
        \item multi-agent models
        \item cell-based models
    \end{itemize}
\end{itemize}
