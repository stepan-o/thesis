When it comes to interpreting real estate market dynamics, drawing conclusions from data can be more of an art than a science.\cite{Brett2009}

\subsection{The study of complex urban systems, modelling and computation} \label{subsec:study_of_complex_urban_systems_modelling_computation}

An increasing amount of aspects of human life can be traced back through diverse digital footprints and, when aggregated, can reveal emerging patterns.\cite{Arribas-Bel2014}
As such forms of human activity as transactions of land and real estate ownership become digitized\cite{TeranetEnterprisesInc.}, a wealth of data becomes captured and available for analysis.
These increasingly comprehensive archives of human behaviour, combined with the exponential growth of computational power, create potential for transformations in such fields as sociology\cite{Lazer2017} and travel behaviour analysis\cite{Chen2016}.

However, when it comes to using these data sources in social studies, along with opportunities there are also challenges present.
For example, these data sources can have issues with the quality of the data, might require a specific set of skills to take advantage of these data sources, or might not be suitable for traditional methods meant for traditional data\cite{Arribas-Bel2014}.
In addition, when it comes to interpreting real estate market dynamics, drawing conclusions from data can be more of an art than a science.\cite{Brett2009}
Therefore, to study the interaction between land development, urban form, and transportation, it would be beneficial to have a system that can facilitate efficient access to linked information describing these phenomena to researchers from a wide range of disciplines and backgrounds.

The study of complex systems, including urban systems such as land development or transportation, is intrinsically tied to modelling and computation.
Various computer-based models allow us to explore and improve our understanding of the behaviour of a complex system.
Thus, systematic analysis and modelling of urban systems intrinsically depends on our ability to represent and model urban regions and urban processes, which in turn is often limited by our computing software and hardware capabilities.

With large amounts of information about urban regions digitalized\cite{TeranetEnterprisesInc.}, computer-based data storage and data display, manipulation, and management systems, such as Geographic Information Systems (GIS) or Relational Database Management Systems (RDBMS), nowadays play an increasingly important role.
These systems become important because they offer researchers easy access to a wide array of diverse data sources and allow efficient workflows to prepare datasets to be used in statistical analysis software, modelling software, machine learning, etc.


TODO: fuzzy diagram of price

Bourne\cite{Bourne1982} defines urban structure as the combination of the following elements:
\begin{itemize}
    \item urban form, or the spatial configuration of fixed elements within the urban region (including land use, buildings, transportation network, etc.)
    \item urban interactions, or the flows of goods, people, information, and money (including commercial activity, real estate market, etc.)
    \item organizing principles, or the relationships between the urban form and interactions (such as travel cost minimization, social status, segregation, planning policy, etc.)
\end{itemize}


One of the possible ways of studying the interaction between the housing market, urban form, and transportation is empirical, such as examining how property values vary with the distance to a transportation facility\cite{Sherry1999}.
However, this type of research requires fine-scale transportation data to be linked to fine-scale housing market data, which presents a challenge since these data sources use different spatial units (as discussed in section~\ref{sec:spatial_relationships}) and are available at different temporal spans (as discussed in section~\ref{sec:termporal_relationships_between_datasets}).
The primary focus of this mater's thesis is the introduction of a knowledge system providing researchers at the University of Toronto Transportation Research Institute (UTTRI) access to the information required to investigate the complex relationship between transportation and land use.

\subsection{Housing market} \label{subsec:housing_market}

\subsection{Models used for transportation and land use} \label{subsec:transportation_land_use_interaction}


Iacono et al.
\cite{Iacono2008} provide an overview of some of the most common frameworks of Land Use and Transport (LUT) models that have been used to model interaction between transportation and land use in urbanized regions.
Despite the difficulty of modelling of every relevant aspect of an urban region, a wide variety of models were produced dealing with the relationship between transportation network growth and changes in land use and the location of economic activity.
The frameworks can broadly be broken into two major approaches to modelling interactions between land use and transport:
\begin{itemize}
    \item "top-down" modeling frameworks, where interaction between transportation networks and location is specified as a set of aggregate relationships.
    These relationships are based on the behaviour of a representative individual, and are usually taken as a mean calculated from a representative sample of the population.
    These models include:
    \begin{itemize}
        \item aggregate models of spatial interaction
        \item econometric models
    \end{itemize}
    \item "bottom-up" microsimulation models, which cover a number of different approaches to representing the dynamics of land use change and travel behaviour through disaggregating the population and simulating the changes.
    These models include:
    \begin{itemize}
        \item activity-based travel models
        \item multi-agent models
        \item cell-based models
    \end{itemize}
\end{itemize}

% old Teranet challenges

To address these challenges, an efficient and modular Python workflow and a relational database for GTHA housing market are introduced with this master's thesis.
This makes data science methodologies a natural fit when trying to get a deeper understanding of the nature of "wicked" urban problems, such as the interaction between land development, urban form, and transportation.

\subsection{Data quality, skill requirements and lack of features} \label{subsec:challenges_quality_skills_features}

Teranet dataset presents an extensive historical record of real estate transactions recorded in the Province of Ontario since the beginning of XIX century.
However, when it comes to using new data sources in social studies, along with opportunities there are also challenges present.
For example, these data sources can have issues with the quality of the data, might require a specific set of skills to take advantage of these data sources, or might not be suitable for traditional methods meant for traditional data\cite{Arribas-Bel2014}, all of which are true in the case of Teranet's dataset.

On top of the issues with Teranet's data mentioned above, the most significant challenge lies in the amount of features provided for each record.
Despite capturing effectively the complete population of real estate transactions recorded in Ontario since 1985, the available version of the dataset includes no information describing anything about each transaction, other than its location in the form of address and coordinates, the registration date, and the consideration amount.
Since is no distinction being made between transactions recorded for residential, commercial, and industrial property types, transaction amounts can vary from several dollars to billions of dollars.
There are no attributes that would allow transactions to be filtered for meaningful analysis and modeling.

\subsection{Size of datasets, computational requirements and file formats} \label{subsec:challenges_size_and_formats}

To address this gap, additional attributes can be joined from various data sources, such as Census, TTS or land use data, based on spatial or temporal relationships.
However, Teranet's dataset has a number of records on the order of $10^6$.
Such data sources as Census tables have the number of fields on the order of $10^2$.
Overall, the number of attributes that could be joined between the various available data sources will probably be on the order of $10^3$.
This fact makes it increasingly impractical to store all the data in one table, due to the memory requirements for storage and processing.

In addition, datasets coming from different sources arrive in different file formats, such as text files, Excel spreadsheets, .csv files, or shapefiles.
Combined with the size of the datasets and the computational requirements of such processing techniques as spatial joins, the facts mentioned above make working with Teranet's difficult due to the hardware and special skill requirements.

\subsection{Complex spatial and temporal relationships between urban data sources} \label{subsec:complex_spatio-temporal_relationships_between_urban_data_sources}

%TODO write subsection

\subsection{The proposed solution: Python-SQL back-end} \label{subsec:proposed_solution_python_sql_backend}

To address this challenge, this master's thesis establishes an organized modular data preparation workflow for Teranet's dataset and other related data sources and organized them into a relational database.
A Relational Database Management System (RDBMS), such as PostgreSQL, will deliver the benefit of organized and optimized data storage, access, and processing;
it is also capable of facilitating the flexibility and modular structure that is dictated by the nature of the data sources and the scope of research activities undertaken by UTTRI .
The data preparation workflow is facilitated via Python in a series of jupyter notebooks described in section \textbf{cite}
%TODO rewrite proposed solution to ML workflow

\subsection{Accessibility as a measure of land use-transportation relationship} \label{subsec:accessibility}

While transportation research started as an isolated field focusing on mobility, researchers recognized that trips are made to access particular destinations\cite{VanLierop2017}.
A concept of accessibility, understood as the ease of reaching rather than simply the ease of moving\cite{Preston2007}, was developed to take into account the location of urban opportunities.
Accessibility is the mediating factor between the location of activities and demand for travel and is discussed further in section~\ref{subsec:accessibility} of the following chapter.

LUT models (discussed in section~\ref{subsec:transportation_land_use_interaction}) operationalize the transportation-land use relationship by incorporating measures of accessibility with the process of locating activities, typically assuming that households wish to locate in areas with higher accessibility to employment, shopping, or entertainment opportunities.
Similarly, firms are assumed to locate in areas with higher accessibility to labour markets.
Land use component is usually integrated into the accessibility measure through congested network travel times.
However, when studying the relationship between transportation facilities and property values, results may vary based on whether travel time or travel distance is used as a measure of accessibility\cite{Sherry1999}.

Accessibility measures the situation of a location relative to other activities or opportunities (work, shopping, etc.) distributed in space and can be an important determinant of price in LUT models where land and floor space markets are considered explicitly\cite{Iacono2008}.
When measuring changes in relative accessibility, it is usually approximated by some measure of access to the transportation network, such as travel time or distance.

Economic decisions made by the households and firms act as one of the links between the tranportation and land use systems.
When choosing a location, households and firms attempt to fulfill as many of their location preferences as possible while facing constraints.
One of the main constraints is the price or rent of the dwelling, related to the income of the household;
another important constraint is travel time, with suitable location being restricted by the travel times and transportation expenses\cite{Moeckel2017}.
