\chapter[Background]{Background information on the land registration system in Ontario} \label{ch:background}

\section{Transportation, urban form, and real estate market} \label{sec:intro}

There is a fundamental link between transportation and urban form;
thus, the nature of development of urban areas directly determines such aspects as travel needs, viability of alternative modes, etc.
Transportation, in turn, influences land development and location choices of people and firms.\textbf{cite?}

As such, the study of interaction between real estate, urban form, and transportation presents opportunities in uncovering the dynamics of urban processes.
However, when it comes to interpreting real estate market data, drawing conclusions can be more of an art than a science.\cite{Brett2009}
As such, to study the interaction between land development, urban form, and transportation, it would be beneficial to have a system that is both comprehensive and easy to access by a wide range of researchers.
The focus of this master thesis is the organization and implementation of such knowledge management system to facilitate future research activities conducted by the University of Toronto Transportation Research Institute (UTTRI).

\section{Real estate market} \label{sec:real_estate_market}

There are many ways to approach the definition of the real estate market: it could refer to various grouping of customers, as often done in marketing context, while terms supply and demand employed by economists characterize the market in terms of by analyzing behaviour of buyers and sellers.
In real estate, product usually refers to property type, and is distinguished by the property type (apartment buildings, offices, warehouses, etc.), interior features (size, layout, quality of design, amenities/services), location attributes (walkability, access to transit, distance to the city centre, etc.), and prices and rents.
Segmentation of the housing sector can be performed by physical characteristics (single family detached/attached/low-rise/mid-rise/high-rise apartments, etc.).
Narrow definition of the market segment helps fine tune the analysis.\cite{Brett2009}
As such, it would be beneficial to relate a range of data sources together into a single database that can yield targeted datasets with custom attributes to facilitate specialized studies.
This task, which involves the collection of data, design, and implementation of a relational database covering the GTHA housing market, is the primary focus of this master's thesis.

TODO: fuzzy diagram of price

\section{ILUTE model system} \label{sec:ilute}

\textit{ILUTE is a comprehensive, integrated model system designed to project the evolution of demographics, land use and travel within an urban region over time.
The ILUTE (Integrated Land Use, Transportation, Environment) model system is an agent-based microsimulation model for the Greater Toronto-Hamilton Area (GTHA) in which disaggregate, process-based models of spatial socio-economic processes are used to evolve the GTHA system state from a known base case to a predicted future year end state in one-year time steps.
The system state is defined in terms of the individual persons, households, dwelling units, firms, etc.
that collectively define the urban region being modeled.
The housing market component of ILUTE evolves the residential location of households over time.
It includes the endogenous supply of housing by type and location, as well as the endogenous determination of sales prices and rents.}\cite{Miller2010}