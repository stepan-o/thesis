\chapter[Background information]{Background information} \label{ch:background}

\section{The role of data in urban studies} \label{sec:role_of_data_in_urban_studies}

\subsection{Complexity of urban systems and "wicked" problems} \label{subsec:complexity_and_wicked_problems}

In her famous 1961 book, Jane Jacobs\cite{Jacobs1961} described a city as "a problem in organized complexity";
since then, many other researchers have remarked that urban systems exhibit complex behaviour\cite{Batty2008, Bettencourt2013}.
Complexity of a system can be defined as a state or quality of being intricate or complicated.
For a system to be complex is not necessarily the same as to be complicated;
complex systems can be simple, i.e.\ governed by a single equation.
Complexity of a system has to do with the intrinsic ability of a system to surprise us with its behaviour;
that the system is hard to understand, despite the mechanics of it being relatively simple.

A little over a decade after Jacobs, Rittel and Webber\cite{Rittel1973} presented a path-breaking conceptualization;
this conceptualization characterized urban planning problems as "wicked" problems: problems which cannot be definitively described and for which it makes no sense to talk of "optimal solutions".
More than 40 years after their original publication, Rittel and Webber's ideas remain relevant to the policy sciences today: there is an intense interest in the nature of "wicked" problems and the complex tasks of identifying their scope, viable responses, and appropriate mechanisms and pathways to improvement\cite{Crowley2017}.
In their paper, Ritter and Webber stated that such "wicked" problems are never "solved", and that the focus instead becomes on iteratively "re-solving" the problems over and over.
This approach resembles the methodologies typically employed for data science projects, where the sequence of steps is iterated over, producing a more meaningful solution on each iteration of the cycle, as defined in such process models as CRISP-DM\cite{Shearer2000}.
This makes data science methodologies a natural fit when trying to get a deeper understanding of the nature of "wicked" urban problems, such as the interaction between land development, urban form, and transportation.

\subsection{Transportation, urban form, and real estate market data} \label{subsec:transportation_urban_form_real_estate_data}

Bourne\cite{Bourne1982} defines urban structure as the combination of the following elements:
\begin{itemize}
    \item urban form, or the spatial configuration of fixed elements within the urban region (including land use, buildings, transportation network, etc.)
    \item urban interactions, or the flows of goods, people, information, and money (including commercial activity, real estate market, etc.)
    \item organizing principles, or the relationships between the urban form and interactions (such as travel cost minimization, social status, segregation, etc.)
\end{itemize}
There is a fundamental link between transportation and urban form;
thus, the nature of development of urban areas directly determines such aspects as travel needs, viability of alternative modes, etc.
Transportation, in turn, influences land development and location choices of people and firms\cite{Miller1999}.
This relationship is sometimes referred to as the transportation-land use "link" or "cycle", emphasising a feedback relationship\cite{Kelly}.
As such, the study of interaction between real estate, urban form, and transportation presents opportunities in uncovering the dynamics of urban processes.
One of the possible ways of studying these dynamics is empirical, such as examining how property values vary with distance to transportation facility\cite{Sherry1999}.

\subsection{The study of complex urban systems, modelling and computation} \label{subsec:study_of_complex_urban_systems_modelling_computation}

An increasing amount of aspects of human life can be traced back through diverse digital footprints and, when aggregated, can reveal emerging patterns.\cite{Arribas-Bel2014}
As such forms of human activity as transactions of land and real estate ownership become digitized\cite{TeranetEnterprisesInc.}, a wealth of data becomes captured and available for analysis.
These increasingly comprehensive archives of human behaviour, combined with the exponential growth of computational power, create potential for transformations in such fields as sociology\cite{Lazer2017}.
However, when it comes to interpreting real estate market dynamics, drawing conclusions from data can be more of an art than a science.\cite{Brett2009}
Therefore, to study the interaction between land development, urban form, and transportation, it would be beneficial to have a system that can facilitate efficient access to linked information describing these phenomena to researchers from a wide range of disciplines and backgrounds.

The study of complex systems, including urban systems such as land development or transportation, is intrinsically tied to modelling and computation.
Various computer-based models allow us to explore and improve our understanding of the behaviour of a complex system.
Thus, systematic analysis and modelling of urban systems intrinsically depends on our ability to represent and model urban regions and urban processes, which in turn is often limited by our computing software and hardware capabilities.

With large amounts of information about urban regions digitalized\cite{TeranetEnterprisesInc.}, computer-based data storage and data display, manipulation, and management systems, such as Geographic Information Systems (GIS) or Relational Database Management Systems (RDBMS), nowadays play an increasingly important role.
These systems become important because they offer researchers easy access to a wide array of diverse data sources and allow efficient workflows to prepare datasets to be used in statistical analysis software, modelling software, machine learning, etc.

\section{Housing market studies} \label{sec:housing_market_studies}

\subsection{Land Use and Transport (LUT) models} \label{subsec:evolution_of_models_of_urban_systems}

The history of treating cities as systems via simulation models of transportation and land use dates back to 1950s\cite{Batty2008}.
Iacono et al.
\cite{Iacono2008} provide an overview of some of the most common frameworks of Land Use and Transport (LUT) models that have been used to model interaction between transportation and land use in urbanized regions.
Despite the difficulty of modelling of every relevant aspect of an urban region, a wide variety of models were produced dealing with the relationship between transportation network growth and changes in land use and the location of economic activity.
The frameworks can broadly be broken into two major approaches to modelling interactions between land use and transport:
\begin{itemize}
    \item "top-down" modeling frameworks, where interaction between transportation networks and location is specified as a set of aggregate relationships.
    These relationships are based on the behaviour of a representative individual, and are usually taken as a mean calculated from a representative sample of the population.
    \item "bottom-up" microsimulation models, which cover a number of different approaches to representing the dynamics of land use change and travel behaviour through disaggregating the population and simulating the changes.
    These models include:
    \begin{itemize}
        \item activity-based travel models
        \item multi-agent models
        \item cell-based models
    \end{itemize}
\end{itemize}

\subsection{Accessibility as a measure of land use-transportation relationship} \label{subsec:accessibility}

LUT models operationalize the transportation-land use relationship by incorporating measures of accessibility with the process of locating activities, typically assuming that households wish to locate in areas with higher accessibility to employment, shopping, or entertainment opportunities.
Similarly, firms are assumed to locate in areas with higher accessibility to labour markets.
Accessibility measures the situation of a location relative to other activities or opportunities (work, shopping, etc.) distributed in space\cite{Iacono2008}.
When measuring changes in relative accessibility, it is usually approximated by some measure of access to the transportation network, such as travel time or distance.
Land use component is usually integrated into the accessibility measure through congested network travel times.
However, when studying the relationship between transportation facilities and property values, results may vary based on whether travel time or travel distance is used as a measure of accessibility\cite{Sherry1999}.

\subsection{Breakdown of an urban region} \label{subsec:breakdown_of_urban_region}

To simulate the changes in accessibility, metropolitan regions are usually broken down into a set of small geographic zones, similar (or in many cases identical) to the set of zones used for regional travel forecasting.
Changes to relative accessibility of a location can thus be estimated as changes in zone-to-zone travel times in a travel network\cite{Iacono2008}.

\subsection{ILUTE model system} \label{subsec:ilute}

\textit{ILUTE is a comprehensive, integrated model system designed to project the evolution of demographics, land use and travel within an urban region over time.
The ILUTE (Integrated Land Use, Transportation, Environment) model system is an agent-based microsimulation model for the Greater Toronto-Hamilton Area (GTHA) in which disaggregate, process-based models of spatial socio-economic processes are used to evolve the GTHA system state from a known base case to a predicted future year end state in one-year time steps.
The system state is defined in terms of the individual persons, households, dwelling units, firms, etc.
that collectively define the urban region being modeled.
The housing market component of ILUTE evolves the residential location of households over time.
It includes the endogenous supply of housing by type and location, as well as the endogenous determination of sales prices and rents.}\cite{Miller2010}

\subsection{Real estate market} \label{subsec:real_estate_market}

In speaking of trade we always explicitly speaking of the concept of a market.
Markets, such as the land and real estate markets, are fundamental to understanding of most spatial socio-economic processes that drive the city's evolution.
There are many ways to approach the definition of the real estate market\cite{Brett2009}: it could refer to various grouping of customers, as often done in marketing context, while terms supply and demand employed by economists characterize the market in terms of by analyzing behaviour of buyers and sellers.
In real estate, product usually refers to property type, and is distinguished by the property type (apartment buildings, offices, warehouses, etc.), interior features (size, layout, quality of design, amenities/services), location attributes (walkability, access to transit, distance to the city centre, etc.), and prices and rents.
Segmentation of the housing sector can be performed by physical characteristics (single family detached/attached/low-rise/mid-rise/high-rise apartments, etc.).
Narrow definition of the market segment helps fine tune the analysis.
As such, it would be beneficial to relate a range of data sources together into a single database that can yield targeted datasets with custom attributes to facilitate specialized studies.
This task, which involves the collection of data, design, and implementation of a relational database covering the GTHA housing market, is the primary focus of this master's thesis.

TODO: fuzzy diagram of price

\section{Information system for housing market data} \label{sec:information_system_for_housing_market_data}

\subsection{Requirements to the information system for housing market data} \label{subsec:requirements_to_information_system}

On one hand, for such information-handling system to be comprehensive, it needs to combine a wide range of data sources describing these systems while maintaining semantic interoperability between these sources.
In the case of land use, transportation, demographic, and real estate data, it means to take into account the varying spatial and temporal scale and resolution between these data sources when joining them together.
At the same time, the system needs to be easily accessible to a wide range of researchers and students;
it also needs to have powerful data processing capacity, to allow working with and performing calculations on large datasets related to real estate and land use.
In addition to that, the system should have a modular structure, have a workflow that is reproducible and modifiable, so that new data sources and new relationships can be added to the system, while maintaining the existing part intact.

All of the requirements listed above present a strong case for the housing market information system to be implemented in a form of a relational database.
Given the size of the datasets and the current research needs, PostgreSQL presents a good option for the database management system that fits all the discussed criteria.
The focus of this master thesis is the organization and implementation of the GTHA housing market database to facilitate future research activities focused on the Longitudinal Analysis of housing sales in the Greater Toronto-Hamilton Area conducted by the University of Toronto Transportation Research Institute (UTTRI).

