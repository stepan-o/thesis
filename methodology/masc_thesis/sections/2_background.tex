\chapter[Background information]{Background information} \label{ch:background}

\section{Chapter 2: transportation and land use, land registration in Canada and Teranet} \label{sec:chapter_2_intro}

This chapter discusses the complex interaction of land use and transportation, provides a brief overview of the history of development of land use-transportation (LUT) models, presents some legal and historical background for Teranet's dataset of real estate transactions and finishes with discussing challenges of working with Teranet's data and the proposed solution.

\section{Land Use and Transport (LUT) models} \label{sec:evolution_of_models_of_urban_systems}

\subsection{Complexity of urban systems and "wicked" problems} \label{subsec:complexity_and_wicked_problems}

In her famous 1961 book, Jane Jacobs\cite{Jacobs1961} described a city as "a problem in organized complexity";
since then, many other researchers have remarked that urban systems exhibit complex behaviour\cite{Batty2008, Bettencourt2013}.
Complexity of a system can be defined as a state or quality of being intricate or complicated.
For a system to be complex is not necessarily the same as to be complicated;
complex systems can be simple, i.e.\ governed by a single equation.
Complexity of a system has to do with the intrinsic ability of a system to surprise us with its behaviour;
that the system is hard to understand, despite the mechanics of it being relatively simple.

In 1973, a little over a decade after Jacobs, Rittel and Webber\cite{Rittel1973} presented a path-breaking conceptualization;
this conceptualization characterized urban planning problems as "wicked" problems: problems which cannot be definitively described and for which it makes no sense to talk of "optimal solutions".
In their paper, Rittel and Webber stated that such "wicked" problems are never "solved", and that the focus instead becomes on iteratively "re-solving" the problems over and over.
More than 40 years after their original publication, Rittel and Webber's ideas remain relevant to the policy sciences today: there is an intense interest in the nature of "wicked" problems and the complex tasks of identifying their scope, viable responses, and appropriate mechanisms and pathways to improvement\cite{Crowley2017}.
Interaction between land use and transportation, which is discussed in the following section, presents a prime example of urban complexities and "wicked" problems.

\subsection{Transportation-land use cycle} \label{subsec:transportation_land_use_cycle}

Among the reasons why transportation and land use interaction is "wicked" are such aspects as pluralism of expectations among stakeholders, institutional complexity in policy making, and scientific uncertainty\cite{Noto2015}.
More importantly, there is a fundamental link between transportation and urban form: urban form has an enormous impact on the type and cost of transportation systems needed to serve residents of a metropolitan area\cite{Kelly1994}.
Transportation, in turn, influences land development and location choices of people and firms, and thus completes the formation of a feedback relationship that Stover and Koepke\cite{Stover1988} referred to as a cycle.
Interconnections between points (activities) in space can be perceived through the medium of the transportation system\cite{Miller1998}

Figure~\ref{fig:idealized_integrated_urban_model} illustrates the complex interactions between land use and transportation system as summarized by Miller, Kriger and Hunt\cite{Miller1998}.

\begin{figure}[hbt!]
    \centering
    \includegraphics[width=0.7\linewidth,trim=0 0 0 0,clip]{miller_idealized_urban_model.png}
    \caption{An Idealized Integrated Urban Model System, adapted from Miller, Kriger and Hunt\cite{Miller1998}.}
    \label{fig:idealized_integrated_urban_model}
\end{figure}

Many different types of models are used in planning, such as demand forecasting models projecting traffic or ridership, or land use models projecting and distributing population and jobs within an area.
At an earlier stage of model development, some analysts argued that there is no significant link between transportation and land use, given the near-ubiquity of the transportation (road) network\cite{Miller1998}.
However, the unprecedented urban growth of the 21st century introduced new challenges for urban systems such as extreme road congestion, equity of access to jobs and services among low-income households, energy scarcity, environmental and GHG impacts from transportation systems and public health impacts of land use patterns\cite{Miller2018b,Moeckel2017}.

It became apparent that these "transport problems" cannot be solved through transportation policies and investment alone, that the physical design of the city at the "macro" and "micro" scale critically interfaces with the demand for and performance of the transportation system.
In addition, to accurately assess the costs and benefits of an expensive long-term transportation infrastructure investment, "feedback" effects of these investments on urban form, land values, property taxes, quality of life, etc.
need to be quantified and included in evaluation and decision making.
Thus, today there is a steadily growing recognition within the urban policy field that the interaction between transportation and land use does exist and does matter\cite{Miller2018b}.

In the context of models, integrated urban models (IUMs) aim to capture the complex relationship between urban systems such as transportation and land use more accurately.
Integrated land use-transportation models combine travel demand forecasting and land use forecasting functions and recognize that the distribution of population and jobs depends, in part, on transportation accessibility.
The reverse is also true, and thus integrated models incorporate feedback relationship between transportation and land use, with economic decisions by households and firms acting as one of the links between the two systems\cite{Miller1998}.

\subsection{Evolution of LUT models} \label{subsec:evolution_of_lut_models}

The history of treating cities as systems via simulation models of transportation and land use dates back to 1950s when General System Theory and Cybernetics came to be applied in the softer social sciences\cite{Batty2008}.
The first operational simulation model that truly integrated land use and transportation is considered to be A Model of Metropolis built in 1964 by Ira S. Lowry for the Pittsburgh region based on economic base theory\cite{Lowry1964}.
It was a highly aggregate model based on theories of spatial interaction, such as the gravity model that was popular in quantitative geography and transportation planning at the time\cite{Bouchard1965}.
Models based on spatial interaction framework continued to be developed through mid-1980s, until developments in random utility theory allowed researchers to describe choices among discrete alternatives, such as the choice of travel mode, and generate models based on the study of disaggregate behaviour\cite{Iacono2008}.

Figure~\ref{fig:lut_model_evolution} provides the general overview of chronological development of LUT models summarized by Iacono\cite{Iacono2008}.

\begin{figure}[hbt!]
    \centering
    \includegraphics[width=0.7\linewidth,trim=0 0 0 0,clip]{lut_models_evolution.png}
    \caption{Chronological development of LUT models summarized by Iacono\cite{Iacono2008}.}
    \label{fig:lut_model_evolution}
\end{figure}

The modeling paradigm has changed fundamentally in the early 1990s along with the advances in computing power and efficiency of data storage.
Urban systems used to be viewed as hierarchical and centrally organized equilibrium structures, or "top-down".
Instead, now they were considered to be structured from the "bottom-up", dynamically retaining their integrity through interactions of numerous microelements\cite{Batty2008}.
A new broad class of LUT models that could fall under the title of "microsimulation" began to be developed.
It included such classes of models as activity-based travel, cell-based models, and multi-agent models, and more recently comprehensive urban microsimulation models that fully reflect the dynamics of changes in the population and the urban environment\cite{Iacono2008}.

"Micro" in the microsimulation implies that the model must be highly disaggregated spatially, socio-economically and in its representation of processes.
"Simulation" implies that the model must be numerical, stochastic, have an explicit time dimension, and "evolve" into the end state rather than "solve for it"\cite{Miller2018c}.
An example of such model, an integrated urban model capable of microsimulating urban demographic evolution, housing markets and travel behaviour over extended periods of time\cite{Miller2018a}, has been developed by the University of Toronto ILUTE team.
This model system and some of the ways for its future improvement are discussed in the following section.

\subsection{ILUTE model system} \label{subsec:ilute}

The Integrated Land Use, Transportation, Environment (ILUTE) model system is an agent-based microsimulation model for greater Toronto-Hamilton area;
it includes such components as land use, activity / travel, urban economics, auto ownership, demographics and emissions / energy use.
It uses disaggregate models of spatial socioeconomic processes to evolve the state of the greater Toronto\textendash Hamilton area from a known base case to a predicted end state in 1-year time steps.
The system state is defined in terms of the individual persons, households, dwelling units, firms, etc.
that collectively define the urban region being modeled\cite{Miller2011}.

ILUTE model simulates the evolution of an urban region's spatial form, demographics, travel behavior and economic structure over time.
Many market are of interest within ILUTE, such as housing, labour, commercial, real estate, etc.) and are modeled via microsimulation.
For example, in the housing market component of ILUTE houses are auctioned off one dwelling at a time to interested bidders in a disaggregate implementation of Martinez' Bid Choice theory\cite{Martinez1992}.

The Housing Market Evolutionary System (HoMES) is the updated housing market module for the ILUTE model system.
HoMES is a disaggregate, agent-based microsimulation of the owner-occupied housing market that evolves the residential location of households over time and includes the endogenous supply of housing by type and location, as well as the endogenous determination of sales prices and rents.

An overview of the framework of housing market supply, demand and clearing mechanisms utilized in HoMES provided by Rosenfield et al.\cite{Rosenfield2013} is presented on figure~\ref{fig:homes_framework}

\begin{figure}[hbt!]
    \centering
    \includegraphics[width=0.99\linewidth,trim=0 0 0 0,clip]{homes_framework.png}
    \caption{Framework of housing market supply, demand and clearing mechanisms used in HoMES module of ILUTE, as summarized by Rosenfield et al.\cite{Rosenfield2013}.}
    \label{fig:homes_framework}
\end{figure}

Among the major barriers to implementation of integrated urban models since their introduction were such aspects as data hungriness and computational requirements\cite{Miller1998}.
However, continuing methodological advances, such as cost-effective High Performance Computing (HPC), detailed GIS-based datasets and machine learning methods, mean that former barriers now represent opportunities for model system development.
In the case of ILUTE, one of the possibilities for further improvement is the use of new data sources to update the housing market model.

As an increasing amount of aspects of human life becomes digitalized, a wealth of new data is produced and can be used to model and analyze dynamics of urban systems\cite{Arribas-Bel2014, Chen2016}.
An example of such digitalization of human activity is the introduction of the Province of Ontario Land Registration Information System (POLARIS) in 1985 by the Government of Ontario\cite{TeranetEnterprisesInc.} that will be discussed in the following section.
Introduction of POLARIS lead to the creation of an extensive dataset of real estate transactions (land registration records) by Teranet Enterprises Inc.
This dataset offers a very fine resolution of housing market data across both time and space, which can be beneficial for updating and testing microsimulation models, but it also presents challenges to work with that will be discussed in section~\ref{sec:challenges} of this chapter;
this chapter concludes with the proposed solution to address the challenges.

\section{System of land registration in Canada, POLARIS and Teranet} \label{sec:system_of_land_registration_polaris_teranet}

\subsection{System of land registration in Canada} \label{subsec:land_reg_system_canada}

According to Chapter 6 of the International Comparative Legal Guide to Real Estate published by the Global Legal Group in 2015\cite{McKean2015}, all land owned in Canada is registered in a public land registry through either a registry system, a land titles system or a combination of both in the applicable province.
The registry system is a public record of documents evidencing transactions affecting land.
In the land titles system, the applicable provincial government determines the quality of the title, and essentially guarantees (within certain important statutory limits) the title to, and interests in, the property.
As of 2015, most common law provinces and territories were using the land titles system or were in the process of converting title from a registry system to a land titles system.

On the purchase and sale of real estate and land, ownership is generally transferred to the buyer when the deed or transfer is registered in the applicable land registry office.
An agreement of purchase and sale must be in writing to be enforceable.
A transfer of ownership is actualised by registering, either physically or electronically (depending on the applicable land registry system), a deed or transfer with the applicable land registry office or land registrar, copies of which can be obtained from the relevant registry office, often electronically.

\subsection{POLARIS: Land registration system of Ontario} \label{subsec:polaris}

Each province and territory in Canada has its own land registry system, whether it is a land titles system, a registry system or a combination of both, with each system having its own rules.
As of 2015, the Province of Ontario has largely converted from registry systems to a land titles system.

In 1985, the Government of Ontario initiated the Province of Ontario Land Registration Information System (POLARIS) pilot project for the purposes of records automation and the conversion from the Registry System to the Land Titles System.
The Land Registration Reform Act (Ontario)\cite{TheGovernmentofOntario1990} was introduced in 1990 to facilitate electronic search and registration of properties and the automation of paper-based records.

POLARIS was built by the Province to house and process electronic land records, which in turn lead to the creation of an extensive dataset of land registration records managed by Teranet Enterprises Inc.
Today, POLARIS is the search/registration and property maintenance system for all automated land records in Ontario.

\subsection{The Teranet-Ontario Partnership} \label{subsec:teranet_ontario}

In 1991, the Government of Ontario established a partnership with Teranet\cite{TeranetEnterprisesInc.2019}, a Toronto-based organization, founded the same year, which provides e-services to legal, real estate, government, financial, and healthcare markets.
The partnership was established to convert Ontario's land registration system to a more modernized electronic title system.
The project involved taking a 200-year-old paper-based system and creating a database with electronic records for more than five million parcels of land.

Teranet converted all qualified Registry properties in Ontario to the Land Titles system and automated existing paper Land Titles parcels.
As a result, 99.9\% of property in Ontario was parcelized and administered under the Land Titles system, which affords a property ownership guarantee by the province.
Teranet fully automated the conversion of millions of paper-based documents and records into the Ontario Electronic Land Registration System (ELRS).

In December 2010, the Government of Ontario extended its exclusive relationship with Teranet by 50 years.
Today, Teranet is the exclusive provider of Ontario's online property search and registration;
it developed, owns and operates the ELRS, and also provides online access to Ontario's Writs System.

\subsection{Challenges of using Teranet's data and the proposed solution} \label{subsec:challenges}

%TODO rewrite section: needs to capture this instead
One of the major attributes missing from the available version of Teranet's dataset is the information about the type of property being transacted, with records of various residential, commercial and industrial properties all being mixed together in the same dataset.
At the same time, Teranet records have timestamps (dates) and location information (x and y coordinates) and thus can be joined to variety of other urban data sources, such as Census demographics, Transportation Tomorrow Survey (TTS) and parcel-level land use information.
It is possible to derive this attribute from additional related sources of information, such as detailed land use or Census demographics.
However, joining these data sources together requires additional considerations, as they use different spatial units and are available at different temporal spans, as will be discussed in chapter~\ref{ch:spatial_and_temporal_relationships_between_urban_data}.


\section{Chapter summary} \label{sec:background_summary}

The fundamental link between transportation and urban form creates a feedback relationship between land development, travel needs, viability of alternative modes, accessibility, and other important characteristics of the urban transportation system.
Numerous "top-down" and "bottom-up" models have been designed to analyze and forecast the behaviour of urban regions and interaction of their transportation and land use systems.
Since urban systems are complex in nature and require "re-solving" over and over, data science process models present a good fit for this task with their iterative structure.

Increased digitization of human activity, such as introduction of POLARIS land registration system by the Government of Ontario in 1985, produce a wealth of new information that can be used to study interaction between land use and transportation at a fine spatial and temporal scale.
Teranet's dataset of real estate transactions presents a wealth of information on the housing market of Ontario and can be used for empirical studies of transportation-land use interaction.
However, along with the opportunities, the new data sources also present new challenges.
Teranet's dataset has some data quality issues that need to be addressed and might require special skills to work with due to its size.
But most importantly, it is very limited in the number of features available for each transaction.

One of the major attributes missing from the available version of Teranet's dataset is the information about the type of property being transacted, with records of various residential, commercial and industrial properties all being mixed together in the same dataset.
At the same time, Teranet records have timestamps (dates) and location information (x and y coordinates) and thus can be joined to variety of other urban data sources, such as Census demographics, Transportation Tomorrow Survey (TTS) and parcel-level land use information.
It is possible to derive this attribute from additional related sources of information, such as detailed land use or Census demographics.
However, joining these data sources together requires additional considerations, as they use different spatial units and are available at different temporal spans, as will be discussed in chapter~\ref{ch:spatial_and_temporal_relationships_between_urban_data}.
