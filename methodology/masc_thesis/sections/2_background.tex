\chapter[Background information]{Background information} \label{ch:background}

\section{Complexity of urban systems, "wicked" problems and data} \label{sec:complexity_and_wicked_problems_and_data}

In her famous 1961 book, Jane Jacobs\cite{Jacobs1961} described a city as "a problem in organized complexity";
since then, many other researchers have remarked that urban systems exhibit complex behaviour\cite{Batty2008, Bettencourt2013}.
Complexity of a system can be defined as a state or quality of being intricate or complicated.
For a system to be complex is not necessarily the same as to be complicated;
complex systems can be simple, i.e.\ governed by a single equation.
Complexity of a system has to do with the intrinsic ability of a system to surprise us with its behaviour;
that the system is hard to understand, despite the mechanics of it being relatively simple.

In 1973, a little over a decade after Jacobs, Rittel and Webber\cite{Rittel1973} presented a path-breaking conceptualization;
this conceptualization characterized urban planning problems as "wicked" problems: problems which cannot be definitively described and for which it makes no sense to talk of "optimal solutions".
More than 40 years after their original publication, Rittel and Webber's ideas remain relevant to the policy sciences today: there is an intense interest in the nature of "wicked" problems and the complex tasks of identifying their scope, viable responses, and appropriate mechanisms and pathways to improvement\cite{Crowley2017}.
In their paper, Rittel and Webber stated that such "wicked" problems are never "solved", and that the focus instead becomes on iteratively "re-solving" the problems over and over.

%TODO: rewrite paragraph
This approach resembles the methodologies typically employed for data science projects, where the sequence of steps is iterated over, producing a more meaningful solution on each iteration of the cycle, as defined by such process models as CRISP-DM\cite{Shearer2000}.
This makes data science methodologies a natural fit when trying to get a deeper understanding of the nature of "wicked" urban problems, such as the interaction between land development, urban form, and transportation.

In addition to that, an increasing amount of aspects of human life becomes digitalized, producing a wealth of data that can be used to model and analyze dynamics of urban systems\cite{Arribas-Bel2014, Chen2016}.
An example of such digitalization of human activity is the introduction of the Province of Ontario Land Registration Information System (POLARIS) in 1985 by the Government of Ontario\cite{TeranetEnterprisesInc.}.
This lead to the creation of an extensive dataset of real estate transactions by Teranet Enterprises Inc.
that offers a very fine resolution of housing market data across both time and space.
This dataset is used as the core of the GTHA housing market database which is introduced with this master's thesis.

\section{Land Use and Transport (LUT) models} \label{sec:evolution_of_models_of_urban_systems}

\subsection{Transportation-land use cycle} \label{subsec:transportation_urban_form_real_estate_data}

Bourne\cite{Bourne1982} defines urban structure as the combination of the following elements:
\begin{itemize}
    \item urban form, or the spatial configuration of fixed elements within the urban region (including land use, buildings, transportation network, etc.)
    \item urban interactions, or the flows of goods, people, information, and money (including commercial activity, real estate market, etc.)
    \item organizing principles, or the relationships between the urban form and interactions (such as travel cost minimization, social status, segregation, etc.)
\end{itemize}
There is a fundamental link between transportation and urban form;
thus, the nature of development of urban areas directly determines such aspects as travel needs, viability of alternative modes, etc.
Transportation, in turn, influences land development and location choices of people and firms\cite{Miller1999}.
This relationship is sometimes referred to as the transportation-land use "link" or "cycle", emphasising a feedback relationship\cite{Kelly}.
As such, the study of interaction between real estate, urban form, and transportation presents opportunities in uncovering the dynamics of urban processes.
One of the possible ways of studying these dynamics is empirical, such as examining how property values vary with distance to transportation facility\cite{Sherry1999}.

\subsection{Approaches to model interaction between transportation and land use} \label{subsec:approaches}

The history of treating cities as systems via simulation models of transportation and land use dates back to 1950s\cite{Batty2008}.
Iacono et al.
\cite{Iacono2008} provide an overview of some of the most common frameworks of Land Use and Transport (LUT) models that have been used to model interaction between transportation and land use in urbanized regions.
Despite the difficulty of modelling of every relevant aspect of an urban region, a wide variety of models were produced dealing with the relationship between transportation network growth and changes in land use and the location of economic activity.
The frameworks can broadly be broken into two major approaches to modelling interactions between land use and transport:
\begin{itemize}
    \item "top-down" modeling frameworks, where interaction between transportation networks and location is specified as a set of aggregate relationships.
    These relationships are based on the behaviour of a representative individual, and are usually taken as a mean calculated from a representative sample of the population.
    \item "bottom-up" microsimulation models, which cover a number of different approaches to representing the dynamics of land use change and travel behaviour through disaggregating the population and simulating the changes.
    These models include:
    \begin{itemize}
        \item activity-based travel models
        \item multi-agent models
        \item cell-based models
    \end{itemize}
\end{itemize}

\subsection{Evolution of LUT models} \label{subsec:evolution_of_lut_models}
%TODO add a section on evolution of LUT models

\subsection{ILUTE model system} \label{subsec:ilute}

\textit{ILUTE is a comprehensive, integrated model system designed to project the evolution of demographics, land use and travel within an urban region over time.
The ILUTE (Integrated Land Use, Transportation, Environment) model system is an agent-based microsimulation model for the Greater Toronto-Hamilton Area (GTHA) in which disaggregate, process-based models of spatial socio-economic processes are used to evolve the GTHA system state from a known base case to a predicted future year end state in one-year time steps.
The system state is defined in terms of the individual persons, households, dwelling units, firms, etc.
that collectively define the urban region being modeled.
The housing market component of ILUTE evolves the residential location of households over time.
It includes the endogenous supply of housing by type and location, as well as the endogenous determination of sales prices and rents.}\cite{Miller2010}
%TODO finish description of ILUTE

\subsection{Accessibility as a measure of land use-transportation relationship} \label{subsec:accessibility}

LUT models operationalize the transportation-land use relationship by incorporating measures of accessibility with the process of locating activities, typically assuming that households wish to locate in areas with higher accessibility to employment, shopping, or entertainment opportunities.
Similarly, firms are assumed to locate in areas with higher accessibility to labour markets.
Accessibility measures the situation of a location relative to other activities or opportunities (work, shopping, etc.) distributed in space\cite{Iacono2008}.
When measuring changes in relative accessibility, it is usually approximated by some measure of access to the transportation network, such as travel time or distance.

Land use component is usually integrated into the accessibility measure through congested network travel times.
However, when studying the relationship between transportation facilities and property values, results may vary based on whether travel time or travel distance is used as a measure of accessibility\cite{Sherry1999}.
To simulate the changes in accessibility, metropolitan regions are usually broken down into a set of small geographic zones, similar (or in many cases identical) to the set of zones used for regional travel forecasting.
Changes to relative accessibility of a location can thus be estimated as changes in zone-to-zone travel times in a travel network\cite{Iacono2008}.

\subsection{Breakdown of an urban region} \label{subsec:breakdown_of_urban_region}

Most urban areas are divided into zones or planning areas on the basis of maintaining similar population sizes and following built or natural boundaries like roads or rivers.
For some research purposes, it could be beneficial to use multiple data sources, such data collected by different surveys, and the smallest spatial scales at which different survey data is available.
For example for census data, a dissemination area is the smallest standard geographic area for which all census data are disseminated.

One of the major sources of demographic and statistical data in Canada are the datasets collected under the national Census program.
Statistics Canada collects every five years the national Census of Canada and disseminates the information by a range of geographic units, also referred to as "Census geography"\cite{MapandDataLibrary2019}.
Census geography follows a certain hierarchy defined by Statistics Canada, with the largest top-level divisions being provinces and territories, lowest-tier divisions to which census data is disseminated are Dissemination Areas (DAs)\cite{StatisticsCanada2018}.
Statistics Canada defines a dissemination area as a small area composed of one or more neighbouring dissemination blocks, roughly uniform in population size targeted from 400 to 700 persons to avoid data suppression\cite{StatisticsCanada2015}.

Another major source of information for most transportation planning studies concerned with Southern Ontario is the Transportation Tomorrow Survey (TTS)\cite{DataManagementGroup2014}, an origin destination travel survey.
The Transportation Tomorrow Survey (TTS), undertaken every five years since 1986, is a cooperative effort by local and provincial government agencies to collect information about urban travel in southern Ontario.
TTS represents a retrospective survey of travel taken by every member (age 11 or over) of the household during the day previous to the telephone or web contact.
The information collected and the method of collection has remained relatively consistent over the seven surveys and includes characteristics of the household, characteristics of each person in the household, and details of the trips taken by each member of the household, including details on any trips taken by transit\cite{Ashby2018}.

The finest level of spatial aggregation available through iDRS is that of the Traffic Zone also referred to as Traffic Analysis Zone (TAZ).
TTS data has been collected for changing TAZ boundaries or in other words, different zone systems due to growing population and expanding extents of the survey in the GTHA region over the years.
To make the TTS data consistent for comparing over all years from 1986 to 2016, the data management group (DMG), the custodian of the dataset derived from TTS, made all surveys available in the 2001 zone system, for convenience of researchers.
Any zone system could have been chosen for that matter.
Not as a rule, but the TAZs roughly follow census tract boundaries, which are slightly bigger than DA boundaries.
Overview of the traffic zones and their boundaries: http://dmg.utoronto.ca/survey-boundary-files#tts

We used the 2001 zone system to model travel times for the GTHA on EMME for all TTS years based on the origin-destination trip data collected in the survey. The travel time data was used to create further transportation accessibility variables.
%TODO add more info about breakdown of an urban region


\section{Challenges of using Teranet's data} \label{sec:challenges}
However, when it comes to using these data sources in social studies, along with opportunities there are also challenges present.
For example, these data sources can have issues with the quality of the data, might require a specific set of skills to take advantage of these data sources, or might not be suitable for traditional methods meant for traditional data\cite{Arribas-Bel2014}.
All of these are true in the case of Teranet's dataset of real estate transactions.
In addition, when it comes to interpreting real estate market dynamics, drawing conclusions from data can be more of an art than a science.\cite{Brett2009}
Therefore, to study the interaction between land development, urban form, and transportation, it would be beneficial to have a system that can facilitate efficient access to linked information describing these phenomena to researchers from a wide range of disciplines and backgrounds.

On top of the issues with Teranet's data mentioned above, the most significant challenge lies in the amount of features provided for each record.
Despite capturing effectively the complete population of real estate transactions recorded in Ontario since 1985, the available version of the dataset includes no information describing anything about each transaction, other than its location in the form of address and coordinates, the registration date, and the consideration amount.
Moreover, the dataset includes records of transactions for all kinds of real estate property with an extremely high variance of prices, and no attributes that would allow transactions to be filtered for meaningful analysis and modeling.

\section{Chapter summary} \label{sec:background_summary}
There is a bunch of different data that can help urban research efforts, but its tricky to work with and disorganized, so we need a database.
%TODO finish chapter summary