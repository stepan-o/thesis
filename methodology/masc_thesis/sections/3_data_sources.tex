\chapter{Data sources to be used in the GTHA housing market database} \label{ch:description_of_data_sources}

\section{Introduction: data sources used in the GTHA housing market database} \label{sec:intro_data_sources}

%TODO add chapter introduction
List data sources used in Teranet database.

\section{Description of data sources} \label{sec:description_of_data_sources}

This section describes different data sources combined into the GTHA housing market database.

\subsection{Teranet's dataset of land registration records} \label{subsec:teranet_description}

%TODO rewrite description for Teranet dataset
Teranet dataset of real estate transactions recorded in the Province of Ontario holds a wealth of information on the housing market of Ontario, but is very limited in the number of available attributes.
The dataset can be augmented by joining additional attributes from various data sources, such as Census or TTS survey, based on spatial and/or temporal relationships.
These relationships are best organized in the form of a relational database based on a database management system, such as PostgreSQL .
Teranet dataset plays an integral part in the proposed GTHA housing market database, design and implementation of which is the primary focus of this Master Thesis.

\subsection{Census of Canada} \label{subsec:census_description}

%TODO finish description of Census
One of the major sources of demographic and statistical data in Canada are the datasets collected under the national Census program.
Statistics Canada collects every five years the national Census of Canada and disseminates the information by a range of geographic units, also referred to as "Census geography"\cite{MapandDataLibrary2019}.
Census geography follows a certain hierarchy defined by Statistics Canada, with the largest top-level divisions being provinces and territories, lowest-tier divisions to which census data is disseminated are Dissemination Areas (DAs)\cite{StatisticsCanada2018}.
Statistics Canada defines a dissemination area as a small area composed of one or more neighbouring dissemination blocks, roughly uniform in population size targeted from 400 to 700 persons to avoid data suppression\cite{StatisticsCanada2015}.

\subsection{Transportation Tomorrow Survey (TTS)} \label{subsec:tts_description}

%TODO finish description of TTS
Another major source of information for most transportation planning studies concerned with Southern Ontario is the Transportation Tomorrow Survey (TTS)\cite{DataManagementGroup2014}, an origin destination travel survey.
The Transportation Tomorrow Survey (TTS), undertaken every five years since 1986, is a cooperative effort by local and provincial government agencies to collect information about urban travel in southern Ontario.
TTS represents a retrospective survey of travel taken by every member (age 11 or over) of the household during the day previous to the telephone or web contact.
The information collected and the method of collection has remained relatively consistent over the seven surveys and includes characteristics of the household, characteristics of each person in the household, and details of the trips taken by each member of the household, including details on any trips taken by transit\cite{Ashby2018}.

The finest level of spatial aggregation available through iDRS is that of the Traffic Zone also referred to as Traffic Analysis Zone (TAZ).
TTS data has been collected for changing TAZ boundaries or in other words, different zone systems due to growing population and expanding extents of the survey in the GTHA region over the years.
To make the TTS data consistent for comparing over all years from 1986 to 2016, the data management group (DMG), the custodian of the dataset derived from TTS, made all surveys available in the 2001 zone system, for convenience of researchers.
Any zone system could have been chosen for that matter.
Not as a rule, but the TAZs roughly follow census tract boundaries, which are slightly bigger than DA boundaries.
Overview of the traffic zones and their boundaries: http://dmg.utoronto.ca/survey-boundary-files#tts

\subsection{DMTI data} \label{subsec:dmti_data}

%TODO add some description of DMTI datasets (EPOI, land use)

\subsection{Detailed land use information from geography department} \label{subsec:detailed_land_use_from_geogrpahy_department}

%TODO add some description of the detailed land use obtained from the geography department

\section{Spatial relationships between datasets} \label{sec:spatial_relationships}

%TODO finish section
This section introduces the spatial relationships between the datasets used in the GTHA housing market database.

\subsection{Breakdown of an urban region} \label{subsec:breakdown_of_urban_region}

Most urban areas are divided into zones or planning areas on the basis of maintaining similar population sizes and following built or natural boundaries like roads or rivers.
For many research purposes, it would be beneficial to use multiple data sources, such as when characterizing the interaction between transportation and land use.
To reveal heterogeneity of the systems it is necessary to use the highest spatial and temporal resolution possible.
This dictates the need for the smallest spatial scales at which different survey data is available.
For example for census data, a dissemination area is the smallest standard geographic area for which all census data are disseminated.

One of the major sources of demographic and statistical data in Canada are the datasets collected under the national Census program.
Statistics Canada collects every five years the national Census of Canada and disseminates the information by a range of geographic units, also referred to as "Census geography"\cite{MapandDataLibrary2019}.
Census geography follows a certain hierarchy defined by Statistics Canada, with the largest top-level divisions being provinces and territories, lowest-tier divisions to which census data is disseminated are Dissemination Areas (DAs)\cite{StatisticsCanada2018}.
Statistics Canada defines a dissemination area as a small area composed of one or more neighbouring dissemination blocks, roughly uniform in population size targeted from 400 to 700 persons to avoid data suppression\cite{StatisticsCanada2015}.

Another major source of information for most transportation planning studies concerned with Southern Ontario is the Transportation Tomorrow Survey (TTS)\cite{DataManagementGroup2014}, an origin destination travel survey.
The Transportation Tomorrow Survey (TTS), undertaken every five years since 1986, is a cooperative effort by local and provincial government agencies to collect information about urban travel in southern Ontario.
TTS represents a retrospective survey of travel taken by every member (age 11 or over) of the household during the day previous to the telephone or web contact.
The information collected and the method of collection has remained relatively consistent over the seven surveys and includes characteristics of the household, characteristics of each person in the household, and details of the trips taken by each member of the household, including details on any trips taken by transit\cite{Ashby2018}.

The finest level of spatial aggregation available through iDRS is that of the Traffic Zone also referred to as Traffic Analysis Zone (TAZ).
TTS data has been collected for changing TAZ boundaries or in other words, different zone systems due to growing population and expanding extents of the survey in the GTHA region over the years.
To make the TTS data consistent for comparing over all years from 1986 to 2016, the data management group (DMG), the custodian of the dataset derived from TTS, made all surveys available in the 2001 zone system, for convenience of researchers.
Any zone system could have been chosen for that matter.
Not as a rule, but the TAZs roughly follow census tract boundaries, which are slightly bigger than DA boundaries.
Overview of the traffic zones and their boundaries: http://dmg.utoronto.ca/survey-boundary-files#tts

We used the 2001 zone system to model travel times for the GTHA on EMME for all TTS years based on the origin-destination trip data collected in the survey.
The travel time data was used to create further transportation accessibility variables.
%TODO streamline info about breakdown of an urban region, move info about spatial units to the next subsection.

\subsection{Spatial units used by the data sources in the GTHA housing market database} \label{subsec:spatial_units_used_in_database}

%TODO write subsection about the spatial units used

\section{Temporal relationships between datasets} \label{sec:termporal_relationships_between_datasets}

\subsection{Temporal scales at which different data sources are available} \label{subsec:temporal_scales}
%TODO write a section about temporal scales

\subsection{Matching temporal scales to facilitate linking datasets} \label{subsec:matching_temporal_scales}
%TODO write a subsection about matching temporal scales

\section{Chapter summary} \label{sec:data_sources_summary}
Different data sources use different spatial and temporal scales, and that's what we are going to address with data prep and the database.
%TODO write chapter summary
